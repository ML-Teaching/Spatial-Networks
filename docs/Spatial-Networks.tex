% Options for packages loaded elsewhere
\PassOptionsToPackage{unicode}{hyperref}
\PassOptionsToPackage{hyphens}{url}
%
\documentclass[
]{article}
\usepackage{amsmath,amssymb}
\usepackage{iftex}
\ifPDFTeX
  \usepackage[T1]{fontenc}
  \usepackage[utf8]{inputenc}
  \usepackage{textcomp} % provide euro and other symbols
\else % if luatex or xetex
  \usepackage{unicode-math} % this also loads fontspec
  \defaultfontfeatures{Scale=MatchLowercase}
  \defaultfontfeatures[\rmfamily]{Ligatures=TeX,Scale=1}
\fi
\usepackage{lmodern}
\ifPDFTeX\else
  % xetex/luatex font selection
\fi
% Use upquote if available, for straight quotes in verbatim environments
\IfFileExists{upquote.sty}{\usepackage{upquote}}{}
\IfFileExists{microtype.sty}{% use microtype if available
  \usepackage[]{microtype}
  \UseMicrotypeSet[protrusion]{basicmath} % disable protrusion for tt fonts
}{}
\makeatletter
\@ifundefined{KOMAClassName}{% if non-KOMA class
  \IfFileExists{parskip.sty}{%
    \usepackage{parskip}
  }{% else
    \setlength{\parindent}{0pt}
    \setlength{\parskip}{6pt plus 2pt minus 1pt}}
}{% if KOMA class
  \KOMAoptions{parskip=half}}
\makeatother
\usepackage{xcolor}
\usepackage[margin=1in]{geometry}
\usepackage{longtable,booktabs,array}
\usepackage{calc} % for calculating minipage widths
% Correct order of tables after \paragraph or \subparagraph
\usepackage{etoolbox}
\makeatletter
\patchcmd\longtable{\par}{\if@noskipsec\mbox{}\fi\par}{}{}
\makeatother
% Allow footnotes in longtable head/foot
\IfFileExists{footnotehyper.sty}{\usepackage{footnotehyper}}{\usepackage{footnote}}
\makesavenoteenv{longtable}
\usepackage{graphicx}
\makeatletter
\def\maxwidth{\ifdim\Gin@nat@width>\linewidth\linewidth\else\Gin@nat@width\fi}
\def\maxheight{\ifdim\Gin@nat@height>\textheight\textheight\else\Gin@nat@height\fi}
\makeatother
% Scale images if necessary, so that they will not overflow the page
% margins by default, and it is still possible to overwrite the defaults
% using explicit options in \includegraphics[width, height, ...]{}
\setkeys{Gin}{width=\maxwidth,height=\maxheight,keepaspectratio}
% Set default figure placement to htbp
\makeatletter
\def\fps@figure{htbp}
\makeatother
\setlength{\emergencystretch}{3em} % prevent overfull lines
\providecommand{\tightlist}{%
  \setlength{\itemsep}{0pt}\setlength{\parskip}{0pt}}
\setcounter{secnumdepth}{5}
\ifLuaTeX
  \usepackage{selnolig}  % disable illegal ligatures
\fi
\usepackage[]{natbib}
\bibliographystyle{plainnat}
\usepackage{bookmark}
\IfFileExists{xurl.sty}{\usepackage{xurl}}{} % add URL line breaks if available
\urlstyle{same}
\hypersetup{
  pdftitle={Introduction aux réseaux spatiaux avec R},
  pdfauthor={Maxime Lenormand},
  hidelinks,
  pdfcreator={LaTeX via pandoc}}

\title{Introduction aux réseaux spatiaux avec R}
\author{\href{https://www.maximelenormand.com/}{Maxime Lenormand}}
\date{2025-08-19}

\begin{document}
\maketitle

{
\setcounter{tocdepth}{2}
\tableofcontents
}
\section*{Préambule}\label{preambule}
\addcontentsline{toc}{section}{Préambule}

Bienvenue dans ce module sur les \textbf{réseaux spatiaux} avec \textbf{R}. Ce module vous
propose une exploration approfondie des réseaux spatiaux et interactions
spatiales en utilisant les packages R \href{https://rtdlm.github.io/TDLM/}{TDLM} et
\href{https://biorgeo.github.io/bioregion/}{bioregion}. Vous apprendrez
à utiliser :

\begin{itemize}
\item
  \textbf{R, RStudio et Shiny} en abordant des
  notions de base pour manipuler des données et pour créer des applications
  interactives simples avec \href{https://shiny.posit.co/}{Shiny} pour visualiser et
  explorer vos réseaux spatiaux de manière dynamique.
\item
  \textbf{TDLM} en comparant rigoureusement différentes lois de distribution des
  déplacements (telles que la loi de gravité ou les opportunités intervenantes)
  et en modélisant les flux de déplacements entre zones géographiques.
\item
  \textbf{bioregion} en appliquant des méthodes de biogéorégionalisation pour
  identifier des unités spatiales homogènes en termes de composition d'espèces,
  en utilisant des algorithmes de clustering hiérarchique, non hiérarchique et
  des méthodes issues de la théorie des réseaux.
\end{itemize}

Ce GitBook est conçu pour être à la fois un \textbf{support théorique} et un
\textbf{guide pratique} : chaque concept sera accompagné d'exemples directement
applicables dans \textbf{R} grâce aux packages
\href{https://rtdlm.github.io/TDLM/}{TDLM} et
\href{https://biorgeo.github.io/bioregion/}{bioregion}.

Avant d'entrer dans le détail des TPs, voici une
introduction générale aux réseaux spatiaux. Elle présente les concepts clés,
le vocabulaire essentiel ainsi que le programme du module.

~

Ce navigateur ne supporte pas les PDF. Télécharger le PDF.

\section{Premiers pas avec R}\label{tp1}

En cours de rédaction

\section{Estimation des flux domicile-travail}\label{tp2}

En cours de rédaction

\section{Bioregionalisation}\label{tp3}

En cours de rédaction

\end{document}
